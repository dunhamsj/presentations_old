\documentclass[11pt]{article}

\begin{document}

A Parameter Study of the SASI in 2D: Newtonian vs. GR Hydrodynamics\newline

Core-collapse supernovae involve multiple branches of physics, all of which interact with each other to produce explosions that populate the universe with elements and produce neutrinos and gravitational waves. One of these branches is hydrodynamics (HD), which is important because some HD instabilities are widely believed to play important roles in the reenergization of the stalled shock. One such instability is the standing accretion shock instability (SASI) (Blondin et al., 2003, ApJ, 548, 971), a phenomenon that is at first best understood using only HD. Using our new code, \texttt{thornado} (Endeve et al., 2019, J. Phys.: Conf. Ser. 1225 012014), we perform a new parameter study of the SASI using (1) Runge-Kutta discontinuous-Galerkin (Cockburn \& Shu, 2001, JSC, 16, 3) methods to obtain high-order accurate results and (2) general relativity (GR). We investigate and discuss growth rates and oscillation frequencies of the SASI in the linear regime and compare amongst the different models. More specifically, we focus on the differences in the evolution of the SASI when using non-relativistic vs. GRHD, and choose values of the parameters (e.g., mass of the proto-neutron star) to reflect typical conditions in the shock-revival phase, as well as to enhance the effects of GR.

\end{document}
