%%%%%%%%%%%%%%%%%%%%%%%%%%%%%%%%%%%%%%%%%
% Beamer Presentation
% LaTeX Template
% Version 1.0 (10/11/12)
%
% This template has been downloaded from:
% http://www.LaTeXTemplates.com
%
% License:
% CC BY-NC-SA 3.0 (http://creativecommons.org/licenses/by-nc-sa/3.0/)
%
%%%%%%%%%%%%%%%%%%%%%%%%%%%%%%%%%%%%%%%%%
%-------------------------------------------------------------------------------
%-------------------------------------------------------------------------------
%	PACKAGES AND THEMES
%-------------------------------------------------------------------------------

\newcommand{\thornado}{\texttt{thornado}}
\newcommand{\poseidon}{\texttt{Poseidon}}
\newcommand{\amrex}{\texttt{AMReX}}

\documentclass{beamer}


\usepackage{subcaption}

\mode<presentation> {

% The Beamer class comes with a number of default slide themes
% which change the colors and layouts of slides. Below this is a list
% of all the themes, uncomment each in turn to see what they look like.

%\usetheme{default}
%\usetheme{AnnArbor}
%\usetheme{Antibes}
%\usetheme{Bergen}
%\usetheme{Berkeley}
%\usetheme{Berlin}
%\usetheme{Boadilla}
%\usetheme{CambridgeUS}
%\usetheme{Copenhagen}
%\usetheme{Darmstadt}
%\usetheme{Dresden}
%\usetheme{Frankfurt}
%\usetheme{Goettingen}
%\usetheme{Hannover}
%\usetheme{Ilmenau}
%\usetheme{JuanLesPins}
%\usetheme{Luebeck}
\usetheme{Madrid}
%\usetheme{Malmoe}
%\usetheme{Marburg}
%\usetheme{Montpellier}
%\usetheme{PaloAlto}
%\usetheme{Pittsburgh}
%\usetheme{Rochester}
%\usetheme{Singapore}
%\usetheme{Szeged}
%\usetheme{Warsaw}

% As well as themes, the Beamer class has a number of color themes
% for any slide theme. Uncomment each of these in turn to see how it
% changes the colors of your current slide theme.

%\usecolortheme{albatross}
%\usecolortheme{beaver}
%\usecolortheme{beetle}
%\usecolortheme{crane}
%\usecolortheme{dolphin}
%\usecolortheme{dove}
%\usecolortheme{fly}
%\usecolortheme{lily}
%\usecolortheme{orchid}
%\usecolortheme{rose}
%\usecolortheme{seagull}
%\usecolortheme{seahorse}
%\usecolortheme{whale}
%\usecolortheme{wolverine}

% To remove the footer line in all slides uncomment this line
%\setbeamertemplate{footline} % To remove the footer line in all slides

% To replace the footer line in all slides with a simple slide count
% uncomment this line
%\setbeamertemplate{footline}[page number]

% To remove the navigation symbols from the bottom of all slides
% uncomment this line
\setbeamertemplate{navigation symbols}{}

} % END \mode<presentation>

% Allows including images
\usepackage{graphicx}

% Allows the use of \toprule, \midrule and \bottomrule in tables
\usepackage{booktabs}

% --- TITLE PAGE ---

\title[Committee Meeting]{Sam's n-th Committee Meeting}

\author{Samuel J. Dunham}
\institute[Vanderbilt University]
{
Vanderbilt University \\
\medskip
\textit{samuel.j.dunham@vanderbilt.edu}
}
\date{August 29, 2022}

\begin{document}

\begin{frame}
\titlepage
\end{frame}

% Table of contents slide, comment this block out to remove it
\begin{frame}
\frametitle{Overview}
\tableofcontents
\end{frame}

% --- PRESENTATION SLIDES ---

% -----------------------
\section{Project Overview}
% -----------------------

\begin{frame}
\frametitle{Project Overview}

\textbf{Primary Goals}
\begin{enumerate}
  \item Develop RKDG solver for 3+1 GRHD equations within \thornado{},
        assuming XCFC
  \item Do 1D CCSN simulations
  \item Investigate role of GR in supernova hydrodynamics,
        in particular the evolution of the SASI
\end{enumerate}

\end{frame}

% -----------------------
\section{Research Update}
% -----------------------

\begin{frame}
\frametitle{Research Update}

\textbf{Code Development (since last time)}
\begin{itemize}
  \item Modified solver to work under XCFC instead of CFC
        (mostly changes to coupling with gravity solver (\poseidon))
  \item Implementing AMR via AMReX (ongoing)
  \item Coupling to \poseidon{} with \amrex{} (ongoing)
  \item Coupling to neutrino transport solver with \amrex{} (ongoing)
\end{itemize}

\end{frame}

\begin{frame}
\frametitle{Ongoing Work}

% The "c" option specifies centered vertical alignment,
% while the "t" option is used for top vertical alignment
\begin{columns}[c]

\column{.45\textwidth} % Left column and width

\textbf{SASI Study}
\begin{itemize}
  \item Investigate effect(s) of GRHD on SASI
  \item Have answered: ``How does GRHD affect SASI?"
        (enough for publication?)
  \item Working on: ``Why does GRHD affect SASI as it does?"/
        "What determines the growth rate?"
\end{itemize}

\column{.5\textwidth} % Right column and width

\begin{figure}
\captionsetup[subfigure]{labelformat=empty}

\centering

% Hack to make figures look nice. Learn proper way to do this

\hspace{-6em}
\begin{subfigure}[c]{0.25\textwidth}
   \includegraphics[width=2.0\textwidth]%
   {fig.GrowthRateComparison_HeatMap.png}
   \caption{}
\end{subfigure}
\hspace{3.5em}
\begin{subfigure}[c]{0.25\textwidth}
   \includegraphics[width=2.0\textwidth]%
   {fig.FrequencyComparison_1D_Rs.png}
   \caption{}
\end{subfigure}

\end{figure}

\end{columns}

\end{frame}

\begin{frame}
\frametitle{Ongoing Work}

% The "c" option specifies centered vertical alignment,
%  while the "t" option is used for top vertical alignment
\begin{columns}[c]

\column{.45\textwidth} % Left column and width

\textbf{1D Adiabatic gravitational collapse problem in parallel with
        multi-level mesh }
\begin{figure}
  \centering
  \includegraphics[width=\textwidth]%
  {fig.AdiabaticCollapse_XCFC_GF_Alpha.png}
  \caption*{Metric fields look smooth.}
\end{figure}

\column{.5\textwidth} % Right column and width

\begin{figure}
  \centering
  \includegraphics[width=\textwidth]%
  {fig.AdiabaticCollapse_XCFC_AF_S.png}
  \caption*{Hydro fields have jumps at refinement boundaries.}
\end{figure}

\end{columns}

\end{frame}

\begin{frame}
\frametitle{Ongoing Work}

\begin{itemize}
  \item Methods paper
  \begin{itemize}
    \item Have a good draft of introduction and physical model section
    \item Ready to start generating data for test-problem section
  \end{itemize}
  \item Coupling to neutrino transport solver
  \begin{itemize}
    \item Testing timestepper (not producing expected results with neutrinos
          turned off)
    \item 1D CCSN simulation
  \end{itemize}
\end{itemize}

\end{frame}

%--------------------
\section{Future Work}
%--------------------

\begin{frame}
\frametitle{Future Work}

\begin{itemize}
  \item Code Development
  \begin{itemize}
    \item Fix issue with tally when using mesh refinement
    \item 3D Hydrodynamics
          ($\sim$two weeks of work, depending on method of choice)
    \item Coupling with Neutrino Transport Module
          ($\sim$one week of work)
  \end{itemize}
  \item Science Targets
  \begin{itemize}
    \item For dissertation
    \begin{itemize}
      \item SASI Study (2D and 3D) ($\sim$two months)
      \item Low-mass 1D CCSN Simulation
            ($\sim$two months pending status of neutrino transport)
      \item High-mass 1D CCSN Simulation
            ($\sim$two months pending status of neutrino transport)
    \end{itemize}
    \item Post-dissertation
    \begin{itemize}
        \item Multi-D CCSN simulations
    \end{itemize}
\end{itemize}
\end{itemize}

\end{frame}

%----------------------------
\section{Future Career Plans}
%----------------------------

\begin{frame}
\frametitle{Future Career Plans}

\centering
\textbf{Graduate in May 2023(?)}

\end{frame}

\begin{frame}
\frametitle{Future Career Plans}

\textbf{Plan A}
\begin{itemize}
  \item 1-2 post-docs (maybe one more year for dissertation)
  \item Faculty position
\end{itemize}

\textbf{Plan B}
\begin{itemize}
  \item 1-2 post-docs (maybe one more year for dissertation)
  \item Staff position at national lab
\end{itemize}

\textbf{Plan C}
\begin{itemize}
  \item Industry
  \begin{itemize}
    \item R\&D for someone (renewable energy would be nice)
    \item Quantitative analyst
    \item IT
  \end{itemize}
\end{itemize}

\end{frame}

\begin{frame}
\frametitle{Considerations}

\begin{itemize}
  \item Location is paramount
        \begin{itemize}
          \item Will post-doc institutions allow remote work?
        \end{itemize}
  \item Want long-term job that can comfortably support a family of $\sim$four
\end{itemize}

\end{frame}

\end{document}
